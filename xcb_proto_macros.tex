%% LyX 1.1 created this file.  For more info, see http://www.lyx.org/.
%% Do not edit unless you really know what you are doing.
\documentclass[english]{article}
\usepackage[T1]{fontenc}
\usepackage{geometry}
\geometry{verbose,letterpaper,tmargin=1in,bmargin=1in,lmargin=1in,rmargin=1in}
\usepackage{babel}
\setlength\parskip{\medskipamount}
\setlength\parindent{0pt}

\makeatletter

%%%%%%%%%%%%%%%%%%%%%%%%%%%%%% LyX specific LaTeX commands.
\providecommand{\LyX}{L\kern-.1667em\lower.25em\hbox{Y}\kern-.125emX\@}

\makeatother
\begin{document}

\title{XCB Protocol Implementation Macro Language}


\author{Jamey Sharp}

\maketitle

\section{Requests and Replies}


\subsection{VOIDREQUEST}

VOIDREQUEST(name, 0 or more PARAMs/FIELDs)

Creates a function named XCB\_<name> returning XCB\_void\_cookie and
accepting whatever parameters are necessary to deliver the given PARAMs
and FIELDs to the X server. Requests created with this macro do not
expect a reply.


\subsection{REQUEST}

REQUEST(name, 0 or more PARAMs, 0 or more REPLYFIELDs)

Creates a function named XCB\_<name> returning XCB\_<name>\_cookie
and accepting whatever parameters are necessary to deliver the given
PARAMs and FIELDs to the X server. Declares the struct XCB\_<name>\_cookie.
Creates a function named XCB\_<name>\_Reply returning a pointer to
x<name>Reply from Xproto.h which forces a cookie returned from XCB\_<name>,
waiting for the response from the server if necessary. If any REPLYFIELDs
are given, they must be quoted.


\subsection{Request Parameters and Fields}

These PARAM and FIELD macros may only appear within the second parameter
to the REQUEST or VOIDREQUEST macros, and must always be quoted.

VALUEPARAM and LISTPARAM macro calls must appear in the order specified
by Appendix B (Protocol Encoding) of the X Window System Protocol
specification. The other macros in this section may appear in any
order, but it is recommended that all PARAMs be in the same order
as given in the protocol specification, so that users of the library
can compare XCB code with the protocol specification.


\subsubsection{PARAM}

PARAM(type, name)

Defines a parameter with a field of the same type.


\subsubsection{VALUEPARAM}

VALUEPARAM(bitmask type, bitmask name, value array name)

Defines a BITMASK/LISTofVALUE parameter pair. The bitmask type should
probably be either CARD16 or CARD32, depending on the specified width
of the bitmask. The value array must be given to the generated function
in the order the X server expects.


\subsubsection{LISTPARAM}

LISTPARAM(element type, list name, length expression)

Defines a LISTofFOO parameter. The length of the list may be given
as any C expression and may reference any of the other fields of this
request.


\subsubsection{LOCALPARAM}

LOCALPARAM(type, name)

Defines a parameter with no associated field. The name can be used
in expressions.


\subsubsection{STRLENFIELD}

STRLENFIELD(string name, field name)

Defines a field which should be filled in with the length of a string
parameter. The string parameter must be defined by a corresponding
LISTPARAM(char, ...), and the field name is available for use in the
expression of a LISTPARAM or EXPRFIELD.


\subsubsection{EXPRFIELD}

EXPRFIELD(field name, expression)

Defines a field which should be filled in with the given expression.
The field name is \textbf{not} available for use in further expressions.


\subsection{REPLYFIELD}

REPLYFIELD(field type, field name, opt length name)

Generates a C pre-processor macro providing access to a variable-length
portion of a reply. If another reply field follows, the length name
must be provided. The length name is the name of a field in the fixed-length
portion of the response which contains the number of elements in this
section.


\section{Utility Macros}


\subsection{Header File Generation}


\subsubsection{\_H and \_C}

For code which should appear only in a header (.h) file or only in
an implementation (.c) file, lines may be prefixed with the \_H and
\_C macros, respectively.


\subsubsection{XCBGEN and ENDXCBGEN}

XCBGEN(header name)\\
ENDXCBGEN()

The first line (aside from comments) of any XCB protocol implementation
should contain a call to the XCBGEN macro, and the last line should
contain a call to the ENDXCBGEN macro. The header name should generally
be the base name of the source file, with all lowercase letters capitalized,
and all other characters converted to underscores ({}``\_''): for
example, xcb\_conn.m4 uses XCB\_CONN here.


\subsection{Indentation}


\subsubsection{INDENT}

INDENT()

Move line indentation to the right 4 spaces.


\subsubsection{UNINDENT}

UNINDENT()

Move line indentation to the left 4 spaces.


\subsubsection{TAB}

TAB()

Indent current line appropriately by inserting spaces.


\subsection{Functions}


\subsubsection{FUNCTION}

FUNCTION(return type and function name, params, body)

Declares a C function, placing the appropriate prototype in the header
file and the complete function in the implementation file.


\subsubsection{STATICFUNCTION}

STATICFUNCTION(return type and function name, params, body)

Declares a static-scope C function. No prototype is placed in the
header file, and the keyword static is applied to the function in
the implementation file.


\subsection{Memory Allocation}

Note that there is no FREE macro; just call free().


\subsubsection{ALLOC}

ALLOC(type, result name, count)

Allocate a block or array of storage with a given name.


\subsubsection{REALLOC}

REALLOC(type, result name, count)

Reallocate a previously-allocated block or array of storage with a
given name, resizing it for the new count.


\subsection{Diversions}


\subsubsection{PUSHDIV}

PUSHDIV(diversion)

Pushes the given diversion onto the diversion stack, switching to
it in the process.


\subsubsection{POPDIV}

POPDIV()

Pops the current diversion off the diversion stack, returning to the
diversion which was in effect before the last PUSHDIV.


\section{Structures and Unions}


\subsection{STRUCT}

STRUCT(name, 1 or more FIELDs)

Typedefs a structure defined to contain the given fields so that it
has the given name.


\subsection{UNION}

UNION(name, 1 or more FIELDs)

Typedefs a union defined to contain the given fields so that it has
the given name.


\subsection{Field Declarations}

These declarations may be used in either STRUCT or UNION definitions.


\subsubsection{FIELD}

FIELD(type, name)

Declares a field of the given type with the given name.


\subsubsection{ARRAYFIELD}

ARRAYFIELD(type, name, quantity)

Declares an array field with the given quantity of elements of the
given type.


\subsubsection{POINTERFIELD}

POINTERFIELD(type, name)

Declares a field with the given name which is a pointer to the given
type.
\end{document}
